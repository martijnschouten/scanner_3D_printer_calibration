%% Generated by Sphinx.
\def\sphinxdocclass{report}
\documentclass[letterpaper,10pt,english]{sphinxmanual}
\ifdefined\pdfpxdimen
   \let\sphinxpxdimen\pdfpxdimen\else\newdimen\sphinxpxdimen
\fi \sphinxpxdimen=.75bp\relax
\ifdefined\pdfimageresolution
    \pdfimageresolution= \numexpr \dimexpr1in\relax/\sphinxpxdimen\relax
\fi
%% let collapsible pdf bookmarks panel have high depth per default
\PassOptionsToPackage{bookmarksdepth=5}{hyperref}

\PassOptionsToPackage{warn}{textcomp}
\usepackage[utf8]{inputenc}
\ifdefined\DeclareUnicodeCharacter
% support both utf8 and utf8x syntaxes
  \ifdefined\DeclareUnicodeCharacterAsOptional
    \def\sphinxDUC#1{\DeclareUnicodeCharacter{"#1}}
  \else
    \let\sphinxDUC\DeclareUnicodeCharacter
  \fi
  \sphinxDUC{00A0}{\nobreakspace}
  \sphinxDUC{2500}{\sphinxunichar{2500}}
  \sphinxDUC{2502}{\sphinxunichar{2502}}
  \sphinxDUC{2514}{\sphinxunichar{2514}}
  \sphinxDUC{251C}{\sphinxunichar{251C}}
  \sphinxDUC{2572}{\textbackslash}
\fi
\usepackage{cmap}
\usepackage[T1]{fontenc}
\usepackage{amsmath,amssymb,amstext}
\usepackage{babel}



\usepackage{tgtermes}
\usepackage{tgheros}
\renewcommand{\ttdefault}{txtt}



\usepackage[Bjarne]{fncychap}
\usepackage{sphinx}

\fvset{fontsize=auto}
\usepackage{geometry}


% Include hyperref last.
\usepackage{hyperref}
% Fix anchor placement for figures with captions.
\usepackage{hypcap}% it must be loaded after hyperref.
% Set up styles of URL: it should be placed after hyperref.
\urlstyle{same}

\addto\captionsenglish{\renewcommand{\contentsname}{Contents:}}

\usepackage{sphinxmessages}
\setcounter{tocdepth}{1}



\title{gcode generator}
\date{Dec 01, 2021}
\release{V0.01}
\author{Martijn Schouten}
\newcommand{\sphinxlogo}{\vbox{}}
\renewcommand{\releasename}{Release}
\makeindex
\begin{document}

\pagestyle{empty}
\sphinxmaketitle
\pagestyle{plain}
\sphinxtableofcontents
\pagestyle{normal}
\phantomsection\label{\detokenize{index::doc}}


\sphinxAtStartPar
The gcode generator consists of two classes. The {\hyperref[\detokenize{index:module-0}]{\sphinxcrossref{\sphinxcode{\sphinxupquote{generator}}}}} class is used to make it easier to generate gcode for basic geometric moves. The {\hyperref[\detokenize{index:module-calibration_pattern}]{\sphinxcrossref{\sphinxcode{\sphinxupquote{calibration\_pattern}}}}} class is used to generate the actual calibration patterns. To use the module to generate a gcode file called “example.g” with a calibration pattern for tools 1,2,3,4 and 5 and with tool 2 as reference use :

\begin{sphinxVerbatim}[commandchars=\\\{\}]
\PYG{k+kn}{from} \PYG{n+nn}{calibration\PYGZus{}pattern} \PYG{k+kn}{import} \PYG{n}{calibartion\PYGZus{}pattern}

\PYG{n}{pattern} \PYG{o}{=} \PYG{n}{calibartion\PYGZus{}pattern}\PYG{p}{(}\PYG{p}{)}

\PYG{n}{tool\PYGZus{}list} \PYG{o}{=} \PYG{p}{[}\PYG{l+m+mi}{1}\PYG{p}{,}\PYG{l+m+mi}{2}\PYG{p}{,}\PYG{l+m+mi}{3}\PYG{p}{,}\PYG{l+m+mi}{4}\PYG{p}{,}\PYG{l+m+mi}{5}\PYG{p}{]}
\PYG{n}{reference\PYGZus{}tool} \PYG{o}{=} \PYG{l+m+mi}{2}
\PYG{n}{lines} \PYG{o}{=} \PYG{n}{pattern}\PYG{o}{.}\PYG{n}{full\PYGZus{}interlocked\PYGZus{}print}\PYG{p}{(}\PYG{n}{tool\PYGZus{}list}\PYG{p}{,}\PYG{n}{reference\PYGZus{}tool}\PYG{p}{,}\PYG{l+s+s2}{\PYGZdq{}}\PYG{l+s+s2}{example.g}\PYG{l+s+s2}{\PYGZdq{}}\PYG{p}{)}
\end{sphinxVerbatim}

\sphinxAtStartPar
By default the code assumes you are using a Diabase H\sphinxhyphen{}series 3D printer. To configure it for another printer or change slicer settings of the gen instance in pattern should be changed. For example to make the code suitable for a dual material printer add the following code before running the \sphinxcode{\sphinxupquote{calibration\_pattern.full\_interlocked\_print()}} function:

\begin{sphinxVerbatim}[commandchars=\\\{\}]
\PYG{n}{pattern}\PYG{o}{.}\PYG{n}{gen}\PYG{o}{.}\PYG{n}{tools} \PYG{o}{=} \PYG{p}{[}\PYG{l+m+mi}{0}\PYG{p}{,}\PYG{l+m+mi}{1}\PYG{p}{]}
\PYG{n}{pattern}\PYG{o}{.}\PYG{n}{gen}\PYG{o}{.}\PYG{n}{standby\PYGZus{}temperatures} \PYG{o}{=} \PYG{p}{[}\PYG{l+m+mi}{175}\PYG{p}{,}\PYG{l+m+mi}{175}\PYG{p}{]}
\PYG{n}{pattern}\PYG{o}{.}\PYG{n}{gen}\PYG{o}{.}\PYG{n}{printing\PYGZus{}temperatures} \PYG{o}{=} \PYG{p}{[}\PYG{l+m+mi}{200}\PYG{p}{,}\PYG{l+m+mi}{200}\PYG{p}{]}
\PYG{n}{pattern}\PYG{o}{.}\PYG{n}{gen}\PYG{o}{.}\PYG{n}{extrusion\PYGZus{}multiplier} \PYG{o}{=} \PYG{p}{[}\PYG{l+m+mf}{1.1}\PYG{p}{,}\PYG{l+m+mf}{1.1}\PYG{p}{]}
\PYG{n}{pattern}\PYG{o}{.}\PYG{n}{gen}\PYG{o}{.}\PYG{n}{retraction\PYGZus{}distance} \PYG{o}{=}\PYG{p}{[}\PYG{l+m+mi}{5}\PYG{p}{,}\PYG{l+m+mi}{5}\PYG{p}{]}
\PYG{n}{pattern}\PYG{o}{.}\PYG{n}{gen}\PYG{o}{.}\PYG{n}{x\PYGZus{}offsets} \PYG{o}{=} \PYG{p}{[}\PYG{l+m+mi}{0}\PYG{p}{,}\PYG{l+m+mi}{0}\PYG{p}{]}
\PYG{n}{pattern}\PYG{o}{.}\PYG{n}{gen}\PYG{o}{.}\PYG{n}{y\PYGZus{}offsets} \PYG{o}{=} \PYG{p}{[}\PYG{l+m+mi}{0}\PYG{p}{,}\PYG{l+m+mi}{0}\PYG{p}{]}
\end{sphinxVerbatim}


\chapter{generator class}
\label{\detokenize{index:module-generator}}\label{\detokenize{index:generator-class}}\index{module@\spxentry{module}!generator@\spxentry{generator}}\index{generator@\spxentry{generator}!module@\spxentry{module}}\phantomsection\label{\detokenize{index:module-0}}\index{module@\spxentry{module}!generator@\spxentry{generator}}\index{generator@\spxentry{generator}!module@\spxentry{module}}\index{generator (class in generator)@\spxentry{generator}\spxextra{class in generator}}

\begin{fulllineitems}
\phantomsection\label{\detokenize{index:generator.generator}}\pysigline{\sphinxbfcode{\sphinxupquote{class\DUrole{w}{  }}}\sphinxcode{\sphinxupquote{generator.}}\sphinxbfcode{\sphinxupquote{generator}}}
\sphinxAtStartPar
Bases: \sphinxcode{\sphinxupquote{object}}

\sphinxAtStartPar
This class can be used to make simple gcode patterns
\index{bed\_temp (generator.generator attribute)@\spxentry{bed\_temp}\spxextra{generator.generator attribute}}

\begin{fulllineitems}
\phantomsection\label{\detokenize{index:generator.generator.bed_temp}}\pysigline{\sphinxbfcode{\sphinxupquote{bed\_temp}}\sphinxbfcode{\sphinxupquote{\DUrole{w}{  }\DUrole{p}{=}\DUrole{w}{  }60}}}
\sphinxAtStartPar
Temperature in degrees celsius to which the bed will be heated

\end{fulllineitems}

\index{enable\_retraction (generator.generator attribute)@\spxentry{enable\_retraction}\spxextra{generator.generator attribute}}

\begin{fulllineitems}
\phantomsection\label{\detokenize{index:generator.generator.enable_retraction}}\pysigline{\sphinxbfcode{\sphinxupquote{enable\_retraction}}\sphinxbfcode{\sphinxupquote{\DUrole{w}{  }\DUrole{p}{=}\DUrole{w}{  }True}}}
\sphinxAtStartPar
Wether or not retraction should be enabled

\end{fulllineitems}

\index{extrude() (generator.generator method)@\spxentry{extrude()}\spxextra{generator.generator method}}

\begin{fulllineitems}
\phantomsection\label{\detokenize{index:generator.generator.extrude}}\pysiglinewithargsret{\sphinxbfcode{\sphinxupquote{extrude}}}{\emph{\DUrole{n}{amount}}}{}
\sphinxAtStartPar
Extrude a specifc amount of filament
\begin{quote}\begin{description}
\item[{Parameters}] \leavevmode
\sphinxAtStartPar
\sphinxstyleliteralstrong{\sphinxupquote{amount}} \textendash{} The amount of filament to extrude

\item[{Returns}] \leavevmode
\sphinxAtStartPar
The gcode to generate the extrusion

\item[{Return type}] \leavevmode
\sphinxAtStartPar
string

\end{description}\end{quote}

\end{fulllineitems}

\index{extrusion\_for\_length() (generator.generator method)@\spxentry{extrusion\_for\_length()}\spxextra{generator.generator method}}

\begin{fulllineitems}
\phantomsection\label{\detokenize{index:generator.generator.extrusion_for_length}}\pysiglinewithargsret{\sphinxbfcode{\sphinxupquote{extrusion\_for\_length}}}{\emph{\DUrole{n}{length}}}{}
\sphinxAtStartPar
Convert a desired line length, to the extrusion volume needed
\begin{quote}\begin{description}
\item[{Parameters}] \leavevmode
\sphinxAtStartPar
\sphinxstyleliteralstrong{\sphinxupquote{volume}} \textendash{} The desired line length

\item[{Returns}] \leavevmode
\sphinxAtStartPar
The extrusion volume needed

\item[{Return type}] \leavevmode
\sphinxAtStartPar
float

\end{description}\end{quote}

\end{fulllineitems}

\index{extrusion\_multiplier (generator.generator attribute)@\spxentry{extrusion\_multiplier}\spxextra{generator.generator attribute}}

\begin{fulllineitems}
\phantomsection\label{\detokenize{index:generator.generator.extrusion_multiplier}}\pysigline{\sphinxbfcode{\sphinxupquote{extrusion\_multiplier}}\sphinxbfcode{\sphinxupquote{\DUrole{w}{  }\DUrole{p}{=}\DUrole{w}{  }{[}1.1, 1.1, 1.1, 1.1, 1.1{]}}}}
\sphinxAtStartPar
Extrusion multiplier used for the tools in the tools list. Should be of the same size as tools.

\end{fulllineitems}

\index{extrusion\_volume\_to\_length() (generator.generator method)@\spxentry{extrusion\_volume\_to\_length()}\spxextra{generator.generator method}}

\begin{fulllineitems}
\phantomsection\label{\detokenize{index:generator.generator.extrusion_volume_to_length}}\pysiglinewithargsret{\sphinxbfcode{\sphinxupquote{extrusion\_volume\_to\_length}}}{\emph{\DUrole{n}{volume}}}{}
\sphinxAtStartPar
Convert a desired volume to be extruded out of the nozzle, to the length of filament that needs to be extruded
\begin{quote}\begin{description}
\item[{Parameters}] \leavevmode
\sphinxAtStartPar
\sphinxstyleliteralstrong{\sphinxupquote{volume}} \textendash{} The desired volume to be extruded out of the nozzle

\item[{Returns}] \leavevmode
\sphinxAtStartPar
The length of filament that needs to be extruded

\item[{Return type}] \leavevmode
\sphinxAtStartPar
float

\end{description}\end{quote}

\end{fulllineitems}

\index{filament\_diameter (generator.generator attribute)@\spxentry{filament\_diameter}\spxextra{generator.generator attribute}}

\begin{fulllineitems}
\phantomsection\label{\detokenize{index:generator.generator.filament_diameter}}\pysigline{\sphinxbfcode{\sphinxupquote{filament\_diameter}}\sphinxbfcode{\sphinxupquote{\DUrole{w}{  }\DUrole{p}{=}\DUrole{w}{  }1.75}}}
\sphinxAtStartPar
Diameter of the used filament in millimeter

\end{fulllineitems}

\index{find\_tools() (generator.generator method)@\spxentry{find\_tools()}\spxextra{generator.generator method}}

\begin{fulllineitems}
\phantomsection\label{\detokenize{index:generator.generator.find_tools}}\pysiglinewithargsret{\sphinxbfcode{\sphinxupquote{find\_tools}}}{\emph{\DUrole{n}{tool\_list}}}{}
\sphinxAtStartPar
Find the tool indexes belonging to a tool number. Usefull to generate the tool\_index parameter of {\hyperref[\detokenize{index:generator.generator.tool_change}]{\sphinxcrossref{\sphinxcode{\sphinxupquote{generator.tool\_change()}}}}}
\begin{quote}\begin{description}
\item[{Parameters}] \leavevmode
\sphinxAtStartPar
\sphinxstyleliteralstrong{\sphinxupquote{tool\_list}} \textendash{} A list of tool numbers.

\item[{Returns}] \leavevmode
\sphinxAtStartPar
A List of tool indices

\item[{Return type}] \leavevmode
\sphinxAtStartPar
list

\end{description}\end{quote}

\end{fulllineitems}

\index{insert\_pause (generator.generator attribute)@\spxentry{insert\_pause}\spxextra{generator.generator attribute}}

\begin{fulllineitems}
\phantomsection\label{\detokenize{index:generator.generator.insert_pause}}\pysigline{\sphinxbfcode{\sphinxupquote{insert\_pause}}\sphinxbfcode{\sphinxupquote{\DUrole{w}{  }\DUrole{p}{=}\DUrole{w}{  }True}}}
\sphinxAtStartPar
Insert a pause between probing the bed and actually printing, during which a piece of paper can be placed on the bed.

\end{fulllineitems}

\index{layer\_height (generator.generator attribute)@\spxentry{layer\_height}\spxextra{generator.generator attribute}}

\begin{fulllineitems}
\phantomsection\label{\detokenize{index:generator.generator.layer_height}}\pysigline{\sphinxbfcode{\sphinxupquote{layer\_height}}\sphinxbfcode{\sphinxupquote{\DUrole{w}{  }\DUrole{p}{=}\DUrole{w}{  }0.2}}}
\sphinxAtStartPar
Used layer height in millimeter

\end{fulllineitems}

\index{line() (generator.generator method)@\spxentry{line()}\spxextra{generator.generator method}}

\begin{fulllineitems}
\phantomsection\label{\detokenize{index:generator.generator.line}}\pysiglinewithargsret{\sphinxbfcode{\sphinxupquote{line}}}{\emph{\DUrole{n}{x}}, \emph{\DUrole{n}{y}}}{}
\sphinxAtStartPar
Generate the gcode for a line from current position with a specific length in x an y
\begin{quote}\begin{description}
\item[{Parameters}] \leavevmode\begin{itemize}
\item {} 
\sphinxAtStartPar
\sphinxstyleliteralstrong{\sphinxupquote{x}} \textendash{} Distance to move in the x direction

\item {} 
\sphinxAtStartPar
\sphinxstyleliteralstrong{\sphinxupquote{y}} \textendash{} Distance to move in the y direction

\end{itemize}

\item[{Returns}] \leavevmode
\sphinxAtStartPar
The gcode to generate the line

\item[{Return type}] \leavevmode
\sphinxAtStartPar
string

\end{description}\end{quote}

\end{fulllineitems}

\index{move() (generator.generator method)@\spxentry{move()}\spxextra{generator.generator method}}

\begin{fulllineitems}
\phantomsection\label{\detokenize{index:generator.generator.move}}\pysiglinewithargsret{\sphinxbfcode{\sphinxupquote{move}}}{\emph{\DUrole{n}{x}}, \emph{\DUrole{n}{y}}}{}
\sphinxAtStartPar
Generate the gcode for a move from current position with a specific length in x an y
\begin{quote}\begin{description}
\item[{Parameters}] \leavevmode\begin{itemize}
\item {} 
\sphinxAtStartPar
\sphinxstyleliteralstrong{\sphinxupquote{x}} \textendash{} Distance to move in the x direction in millimeter

\item {} 
\sphinxAtStartPar
\sphinxstyleliteralstrong{\sphinxupquote{y}} \textendash{} Distance to move in the y direction in millimeter

\end{itemize}

\item[{Returns}] \leavevmode
\sphinxAtStartPar
The gcode to generate the move

\item[{Return type}] \leavevmode
\sphinxAtStartPar
string

\end{description}\end{quote}

\end{fulllineitems}

\index{move\_to() (generator.generator method)@\spxentry{move\_to()}\spxextra{generator.generator method}}

\begin{fulllineitems}
\phantomsection\label{\detokenize{index:generator.generator.move_to}}\pysiglinewithargsret{\sphinxbfcode{\sphinxupquote{move\_to}}}{\emph{\DUrole{n}{x}}, \emph{\DUrole{n}{y}}}{}
\sphinxAtStartPar
Generate the gcode for a move from the current position to a specific coordinate
\begin{quote}\begin{description}
\item[{Parameters}] \leavevmode\begin{itemize}
\item {} 
\sphinxAtStartPar
\sphinxstyleliteralstrong{\sphinxupquote{x}} \textendash{} The x value of the coordinate to move towards in millimeter

\item {} 
\sphinxAtStartPar
\sphinxstyleliteralstrong{\sphinxupquote{y}} \textendash{} The y value of the coordinate to move towards in millimeter

\end{itemize}

\item[{Returns}] \leavevmode
\sphinxAtStartPar
The gcode to generate the move

\item[{Return type}] \leavevmode
\sphinxAtStartPar
string

\end{description}\end{quote}

\end{fulllineitems}

\index{nozzle\_diameters (generator.generator attribute)@\spxentry{nozzle\_diameters}\spxextra{generator.generator attribute}}

\begin{fulllineitems}
\phantomsection\label{\detokenize{index:generator.generator.nozzle_diameters}}\pysigline{\sphinxbfcode{\sphinxupquote{nozzle\_diameters}}\sphinxbfcode{\sphinxupquote{\DUrole{w}{  }\DUrole{p}{=}\DUrole{w}{  }{[}0.4, 0.4, 0.4, 0.4, 0.4{]}}}}
\sphinxAtStartPar
Diameter of the nozzle of the tools in the tools list in millimeter. Should be of the same size as tools.

\end{fulllineitems}

\index{print\_speed (generator.generator attribute)@\spxentry{print\_speed}\spxextra{generator.generator attribute}}

\begin{fulllineitems}
\phantomsection\label{\detokenize{index:generator.generator.print_speed}}\pysigline{\sphinxbfcode{\sphinxupquote{print\_speed}}\sphinxbfcode{\sphinxupquote{\DUrole{w}{  }\DUrole{p}{=}\DUrole{w}{  }30}}}
\sphinxAtStartPar
Used printing speed in mm/s

\end{fulllineitems}

\index{printing\_temperatures (generator.generator attribute)@\spxentry{printing\_temperatures}\spxextra{generator.generator attribute}}

\begin{fulllineitems}
\phantomsection\label{\detokenize{index:generator.generator.printing_temperatures}}\pysigline{\sphinxbfcode{\sphinxupquote{printing\_temperatures}}\sphinxbfcode{\sphinxupquote{\DUrole{w}{  }\DUrole{p}{=}\DUrole{w}{  }{[}200, 200, 200, 200, 200{]}}}}
\sphinxAtStartPar
Printing temperature used for the tools in the tools list in degrees Celsius. Should be of the same size as tools.

\end{fulllineitems}

\index{quarter\_turn() (generator.generator method)@\spxentry{quarter\_turn()}\spxextra{generator.generator method}}

\begin{fulllineitems}
\phantomsection\label{\detokenize{index:generator.generator.quarter_turn}}\pysiglinewithargsret{\sphinxbfcode{\sphinxupquote{quarter\_turn}}}{\emph{\DUrole{n}{x}}, \emph{\DUrole{n}{y}}, \emph{\DUrole{n}{clockwise}}}{}
\sphinxAtStartPar
Generate the gcode for a quarter turn from the currrent position to a new position at a specified distance in x and y
\begin{quote}\begin{description}
\item[{Parameters}] \leavevmode\begin{itemize}
\item {} 
\sphinxAtStartPar
\sphinxstyleliteralstrong{\sphinxupquote{x}} \textendash{} Distance to move in the x direction in millimeter

\item {} 
\sphinxAtStartPar
\sphinxstyleliteralstrong{\sphinxupquote{y}} \textendash{} Distance to move in the y direction in millimeter

\item {} 
\sphinxAtStartPar
\sphinxstyleliteralstrong{\sphinxupquote{clockwise}} \textendash{} If the turn should be clockwise or counter clockwise

\end{itemize}

\item[{Returns}] \leavevmode
\sphinxAtStartPar
The gcode to generate the quarter\_turn

\item[{Return type}] \leavevmode
\sphinxAtStartPar
string

\end{description}\end{quote}

\end{fulllineitems}

\index{reretract() (generator.generator method)@\spxentry{reretract()}\spxextra{generator.generator method}}

\begin{fulllineitems}
\phantomsection\label{\detokenize{index:generator.generator.reretract}}\pysiglinewithargsret{\sphinxbfcode{\sphinxupquote{reretract}}}{}{}
\sphinxAtStartPar
Generate the gcode for a reverse retraction in case retraction is enabled
\begin{quote}\begin{description}
\item[{Returns}] \leavevmode
\sphinxAtStartPar
The gcode to generate the reverse retraction

\item[{Return type}] \leavevmode
\sphinxAtStartPar
string

\end{description}\end{quote}

\end{fulllineitems}

\index{retract() (generator.generator method)@\spxentry{retract()}\spxextra{generator.generator method}}

\begin{fulllineitems}
\phantomsection\label{\detokenize{index:generator.generator.retract}}\pysiglinewithargsret{\sphinxbfcode{\sphinxupquote{retract}}}{}{}
\sphinxAtStartPar
Generate the gcode for a retraction in case retraction is enabled
\begin{quote}\begin{description}
\item[{Returns}] \leavevmode
\sphinxAtStartPar
The gcode to generate the retraction

\item[{Return type}] \leavevmode
\sphinxAtStartPar
string

\end{description}\end{quote}

\end{fulllineitems}

\index{retraction\_distance (generator.generator attribute)@\spxentry{retraction\_distance}\spxextra{generator.generator attribute}}

\begin{fulllineitems}
\phantomsection\label{\detokenize{index:generator.generator.retraction_distance}}\pysigline{\sphinxbfcode{\sphinxupquote{retraction\_distance}}\sphinxbfcode{\sphinxupquote{\DUrole{w}{  }\DUrole{p}{=}\DUrole{w}{  }{[}5, 5, 5, 5, 5{]}}}}
\sphinxAtStartPar
Retraction distance used for the tools in the tools list. Should be of the same size as tools.

\end{fulllineitems}

\index{retraction\_speed (generator.generator attribute)@\spxentry{retraction\_speed}\spxextra{generator.generator attribute}}

\begin{fulllineitems}
\phantomsection\label{\detokenize{index:generator.generator.retraction_speed}}\pysigline{\sphinxbfcode{\sphinxupquote{retraction\_speed}}\sphinxbfcode{\sphinxupquote{\DUrole{w}{  }\DUrole{p}{=}\DUrole{w}{  }80}}}
\sphinxAtStartPar
Retraction speed in mm/s used by all tools

\end{fulllineitems}

\index{rotate() (generator.generator method)@\spxentry{rotate()}\spxextra{generator.generator method}}

\begin{fulllineitems}
\phantomsection\label{\detokenize{index:generator.generator.rotate}}\pysiglinewithargsret{\sphinxbfcode{\sphinxupquote{rotate}}}{\emph{\DUrole{n}{x\_cor}}, \emph{\DUrole{n}{y\_cor}}}{}
\sphinxAtStartPar
Rotate a coordinate around the center of the print
\begin{quote}\begin{description}
\item[{Parameters}] \leavevmode\begin{itemize}
\item {} 
\sphinxAtStartPar
\sphinxstyleliteralstrong{\sphinxupquote{x\_cor}} \textendash{} The x value of the coordinate to rotate around the center

\item {} 
\sphinxAtStartPar
\sphinxstyleliteralstrong{\sphinxupquote{y\_cor}} \textendash{} The y value of the coordinate to rotate around the center

\end{itemize}

\item[{Returns}] \leavevmode
\sphinxAtStartPar
The rotated coordinate

\item[{Return type}] \leavevmode
\sphinxAtStartPar
list

\end{description}\end{quote}

\end{fulllineitems}

\index{rotate\_around\_origin() (generator.generator method)@\spxentry{rotate\_around\_origin()}\spxextra{generator.generator method}}

\begin{fulllineitems}
\phantomsection\label{\detokenize{index:generator.generator.rotate_around_origin}}\pysiglinewithargsret{\sphinxbfcode{\sphinxupquote{rotate\_around\_origin}}}{\emph{\DUrole{n}{x\_cor}}, \emph{\DUrole{n}{y\_cor}}}{}
\sphinxAtStartPar
Rotate a coordinate around the origin (0,0)
\begin{quote}\begin{description}
\item[{Parameters}] \leavevmode\begin{itemize}
\item {} 
\sphinxAtStartPar
\sphinxstyleliteralstrong{\sphinxupquote{x\_cor}} \textendash{} The x value of the coordinate to rotate around (0,0)

\item {} 
\sphinxAtStartPar
\sphinxstyleliteralstrong{\sphinxupquote{y\_cor}} \textendash{} The y value of the coordinate to rotate around (0,0)

\end{itemize}

\item[{Returns}] \leavevmode
\sphinxAtStartPar
The rotated coordinate

\item[{Return type}] \leavevmode
\sphinxAtStartPar
list

\end{description}\end{quote}

\end{fulllineitems}

\index{rotation (generator.generator attribute)@\spxentry{rotation}\spxextra{generator.generator attribute}}

\begin{fulllineitems}
\phantomsection\label{\detokenize{index:generator.generator.rotation}}\pysigline{\sphinxbfcode{\sphinxupquote{rotation}}\sphinxbfcode{\sphinxupquote{\DUrole{w}{  }\DUrole{p}{=}\DUrole{w}{  }0.2617993877991494}}}
\sphinxAtStartPar
Rotation of the structure in radians

\end{fulllineitems}

\index{square() (generator.generator method)@\spxentry{square()}\spxextra{generator.generator method}}

\begin{fulllineitems}
\phantomsection\label{\detokenize{index:generator.generator.square}}\pysiglinewithargsret{\sphinxbfcode{\sphinxupquote{square}}}{\emph{\DUrole{n}{x}}, \emph{\DUrole{n}{y}}, \emph{\DUrole{n}{clockwise}}}{}
\sphinxAtStartPar
Generate the gcode for square with a specific size and the bottom left corner at the current position
\begin{quote}\begin{description}
\item[{Parameters}] \leavevmode\begin{itemize}
\item {} 
\sphinxAtStartPar
\sphinxstyleliteralstrong{\sphinxupquote{x}} \textendash{} The size of the square in the x direction

\item {} 
\sphinxAtStartPar
\sphinxstyleliteralstrong{\sphinxupquote{y}} \textendash{} The size of the square in the y direction

\item {} 
\sphinxAtStartPar
\sphinxstyleliteralstrong{\sphinxupquote{clockwise}} \textendash{} Boolean indicating if the square should be printed clockwise or counter clockwise

\end{itemize}

\item[{Returns}] \leavevmode
\sphinxAtStartPar
The gcode to generate the square

\item[{Return type}] \leavevmode
\sphinxAtStartPar
string

\end{description}\end{quote}

\end{fulllineitems}

\index{standby\_temperatures (generator.generator attribute)@\spxentry{standby\_temperatures}\spxextra{generator.generator attribute}}

\begin{fulllineitems}
\phantomsection\label{\detokenize{index:generator.generator.standby_temperatures}}\pysigline{\sphinxbfcode{\sphinxupquote{standby\_temperatures}}\sphinxbfcode{\sphinxupquote{\DUrole{w}{  }\DUrole{p}{=}\DUrole{w}{  }{[}175, 175, 175, 175, 175{]}}}}
\sphinxAtStartPar
Standby temperature used for the tools in the tools list in degrees Celsius. Should be of the same size as tools.

\end{fulllineitems}

\index{starting\_code() (generator.generator method)@\spxentry{starting\_code()}\spxextra{generator.generator method}}

\begin{fulllineitems}
\phantomsection\label{\detokenize{index:generator.generator.starting_code}}\pysiglinewithargsret{\sphinxbfcode{\sphinxupquote{starting\_code}}}{\emph{\DUrole{n}{tool\_list\_indexes}}}{}
\sphinxAtStartPar
Generate the gcode to start a print. This includes things as homing, heating up the bed and heating up tools
\begin{quote}\begin{description}
\item[{Parameters}] \leavevmode
\sphinxAtStartPar
\sphinxstyleliteralstrong{\sphinxupquote{tool\_list\_indexes}} \textendash{} List of tool indices that should be heated. The indices will be used to select a tool in {\hyperref[\detokenize{index:generator.generator.tools}]{\sphinxcrossref{\sphinxcode{\sphinxupquote{generator.tools}}}}}

\item[{Returns}] \leavevmode
\sphinxAtStartPar
The starting gcode

\item[{Return type}] \leavevmode
\sphinxAtStartPar
string

\end{description}\end{quote}

\end{fulllineitems}

\index{stop\_code() (generator.generator method)@\spxentry{stop\_code()}\spxextra{generator.generator method}}

\begin{fulllineitems}
\phantomsection\label{\detokenize{index:generator.generator.stop_code}}\pysiglinewithargsret{\sphinxbfcode{\sphinxupquote{stop\_code}}}{}{}
\sphinxAtStartPar
Generate the gcode to stop a print. This includes things as moving up the bed and turning off heaters
\begin{quote}\begin{description}
\item[{Returns}] \leavevmode
\sphinxAtStartPar
The stopping gcode

\item[{Return type}] \leavevmode
\sphinxAtStartPar
string

\end{description}\end{quote}

\end{fulllineitems}

\index{tool\_change() (generator.generator method)@\spxentry{tool\_change()}\spxextra{generator.generator method}}

\begin{fulllineitems}
\phantomsection\label{\detokenize{index:generator.generator.tool_change}}\pysiglinewithargsret{\sphinxbfcode{\sphinxupquote{tool\_change}}}{\emph{\DUrole{n}{tool\_index}}}{}
\sphinxAtStartPar
Generate the gcode to perform a tool change
\begin{quote}\begin{description}
\item[{Parameters}] \leavevmode
\sphinxAtStartPar
\sphinxstyleliteralstrong{\sphinxupquote{tool\_list\_indexes}} \textendash{} Tool indices of the tool that should be selected. The indice will be used to select a tool in {\hyperref[\detokenize{index:generator.generator.tools}]{\sphinxcrossref{\sphinxcode{\sphinxupquote{generator.tools}}}}}

\item[{Returns}] \leavevmode
\sphinxAtStartPar
The gcode for the tool change

\item[{Return type}] \leavevmode
\sphinxAtStartPar
string

\end{description}\end{quote}

\end{fulllineitems}

\index{tools (generator.generator attribute)@\spxentry{tools}\spxextra{generator.generator attribute}}

\begin{fulllineitems}
\phantomsection\label{\detokenize{index:generator.generator.tools}}\pysigline{\sphinxbfcode{\sphinxupquote{tools}}\sphinxbfcode{\sphinxupquote{\DUrole{w}{  }\DUrole{p}{=}\DUrole{w}{  }{[}1, 2, 3, 4, 5{]}}}}
\sphinxAtStartPar
List containing the tool numbers of the tools of the 3D printer

\end{fulllineitems}

\index{u\_turn() (generator.generator method)@\spxentry{u\_turn()}\spxextra{generator.generator method}}

\begin{fulllineitems}
\phantomsection\label{\detokenize{index:generator.generator.u_turn}}\pysiglinewithargsret{\sphinxbfcode{\sphinxupquote{u\_turn}}}{\emph{\DUrole{n}{x}}, \emph{\DUrole{n}{y}}, \emph{\DUrole{n}{clockwise}}}{}
\sphinxAtStartPar
Generate the gcode for a u turn from the currrent position to a new position at a specified distance in x and y
\begin{quote}\begin{description}
\item[{Parameters}] \leavevmode\begin{itemize}
\item {} 
\sphinxAtStartPar
\sphinxstyleliteralstrong{\sphinxupquote{x}} \textendash{} Distance to move in the x direction in millimeter

\item {} 
\sphinxAtStartPar
\sphinxstyleliteralstrong{\sphinxupquote{y}} \textendash{} Distance to move in the y direction in millimeter

\item {} 
\sphinxAtStartPar
\sphinxstyleliteralstrong{\sphinxupquote{clockwise}} \textendash{} Boolean indicating if the turn should be clockwise or counter clockwise

\end{itemize}

\item[{Returns}] \leavevmode
\sphinxAtStartPar
The gcode to generate the u turn

\item[{Return type}] \leavevmode
\sphinxAtStartPar
string

\end{description}\end{quote}

\end{fulllineitems}

\index{x\_center (generator.generator attribute)@\spxentry{x\_center}\spxextra{generator.generator attribute}}

\begin{fulllineitems}
\phantomsection\label{\detokenize{index:generator.generator.x_center}}\pysigline{\sphinxbfcode{\sphinxupquote{x\_center}}\sphinxbfcode{\sphinxupquote{\DUrole{w}{  }\DUrole{p}{=}\DUrole{w}{  }0}}}
\sphinxAtStartPar
Location where the center of the printed structure will be (in mm)

\end{fulllineitems}

\index{x\_offsets (generator.generator attribute)@\spxentry{x\_offsets}\spxextra{generator.generator attribute}}

\begin{fulllineitems}
\phantomsection\label{\detokenize{index:generator.generator.x_offsets}}\pysigline{\sphinxbfcode{\sphinxupquote{x\_offsets}}\sphinxbfcode{\sphinxupquote{\DUrole{w}{  }\DUrole{p}{=}\DUrole{w}{  }{[}0, 0, 0, 0, 0{]}}}}
\sphinxAtStartPar
List with additional offsets in the x direction (mm) given to all x moves of the tools in the tools list. Should be of the same size as tools.

\end{fulllineitems}

\index{y\_center (generator.generator attribute)@\spxentry{y\_center}\spxextra{generator.generator attribute}}

\begin{fulllineitems}
\phantomsection\label{\detokenize{index:generator.generator.y_center}}\pysigline{\sphinxbfcode{\sphinxupquote{y\_center}}\sphinxbfcode{\sphinxupquote{\DUrole{w}{  }\DUrole{p}{=}\DUrole{w}{  }0}}}
\sphinxAtStartPar
Location where the center of the printed structure will be (in mm)

\end{fulllineitems}

\index{y\_offsets (generator.generator attribute)@\spxentry{y\_offsets}\spxextra{generator.generator attribute}}

\begin{fulllineitems}
\phantomsection\label{\detokenize{index:generator.generator.y_offsets}}\pysigline{\sphinxbfcode{\sphinxupquote{y\_offsets}}\sphinxbfcode{\sphinxupquote{\DUrole{w}{  }\DUrole{p}{=}\DUrole{w}{  }{[}0, 0, 0, 0, 0{]}}}}
\sphinxAtStartPar
List with additional offsets in the y direction (mm) given to all y moves of the tools in the tools list. Should be of the same size as tools.

\end{fulllineitems}

\index{z\_hop (generator.generator attribute)@\spxentry{z\_hop}\spxextra{generator.generator attribute}}

\begin{fulllineitems}
\phantomsection\label{\detokenize{index:generator.generator.z_hop}}\pysigline{\sphinxbfcode{\sphinxupquote{z\_hop}}\sphinxbfcode{\sphinxupquote{\DUrole{w}{  }\DUrole{p}{=}\DUrole{w}{  }0.5}}}
\sphinxAtStartPar
Wether or not a z\sphinxhyphen{}hop should be performed during a retraction

\end{fulllineitems}

\index{z\_offset (generator.generator attribute)@\spxentry{z\_offset}\spxextra{generator.generator attribute}}

\begin{fulllineitems}
\phantomsection\label{\detokenize{index:generator.generator.z_offset}}\pysigline{\sphinxbfcode{\sphinxupquote{z\_offset}}\sphinxbfcode{\sphinxupquote{\DUrole{w}{  }\DUrole{p}{=}\DUrole{w}{  }0.15}}}
\sphinxAtStartPar
Additional offset in the z direction given to all z moves in millimeter. Allows compensating for printers improperly calibrated in the z direction

\end{fulllineitems}


\end{fulllineitems}



\chapter{calibration\_pattern class}
\label{\detokenize{index:calibration-pattern-class}}\phantomsection\label{\detokenize{index:module-calibration_pattern}}\index{module@\spxentry{module}!calibration\_pattern@\spxentry{calibration\_pattern}}\index{calibration\_pattern@\spxentry{calibration\_pattern}!module@\spxentry{module}}\phantomsection\label{\detokenize{index:module-calibartion_pattern}}\index{module@\spxentry{module}!calibartion\_pattern@\spxentry{calibartion\_pattern}}\index{calibartion\_pattern@\spxentry{calibartion\_pattern}!module@\spxentry{module}}\index{calibration\_pattern (class in calibration\_pattern)@\spxentry{calibration\_pattern}\spxextra{class in calibration\_pattern}}

\begin{fulllineitems}
\phantomsection\label{\detokenize{index:calibration_pattern.calibration_pattern}}\pysigline{\sphinxbfcode{\sphinxupquote{class\DUrole{w}{  }}}\sphinxcode{\sphinxupquote{calibration\_pattern.}}\sphinxbfcode{\sphinxupquote{calibration\_pattern}}}
\sphinxAtStartPar
Bases: \sphinxcode{\sphinxupquote{object}}
\index{differential\_interlocked\_reference\_pattern() (calibration\_pattern.calibration\_pattern method)@\spxentry{differential\_interlocked\_reference\_pattern()}\spxextra{calibration\_pattern.calibration\_pattern method}}

\begin{fulllineitems}
\phantomsection\label{\detokenize{index:calibration_pattern.calibration_pattern.differential_interlocked_reference_pattern}}\pysiglinewithargsret{\sphinxbfcode{\sphinxupquote{differential\_interlocked\_reference\_pattern}}}{\emph{\DUrole{n}{x\_start}}, \emph{\DUrole{n}{y\_start}}, \emph{\DUrole{n}{direction}}}{}
\sphinxAtStartPar
Generate the gcode to print one side (the reference side) of two interlocked patterns, one going up and one going down
\begin{quote}\begin{description}
\item[{Parameters}] \leavevmode\begin{itemize}
\item {} 
\sphinxAtStartPar
\sphinxstyleliteralstrong{\sphinxupquote{x\_start}} \textendash{} The x location of the bottom left corner of the pattern

\item {} 
\sphinxAtStartPar
\sphinxstyleliteralstrong{\sphinxupquote{y\_start}} \textendash{} The y location of the bottom left corner of the pattern

\item {} 
\sphinxAtStartPar
\sphinxstyleliteralstrong{\sphinxupquote{direction}} \textendash{} The direction the pattern should be printed in. Options are: ‘+y’,’+x’

\end{itemize}

\item[{Returns}] \leavevmode
\sphinxAtStartPar
The gcode to generate the pattern

\item[{Return type}] \leavevmode
\sphinxAtStartPar
string

\end{description}\end{quote}

\end{fulllineitems}

\index{differential\_interlocked\_signal\_pattern() (calibration\_pattern.calibration\_pattern method)@\spxentry{differential\_interlocked\_signal\_pattern()}\spxextra{calibration\_pattern.calibration\_pattern method}}

\begin{fulllineitems}
\phantomsection\label{\detokenize{index:calibration_pattern.calibration_pattern.differential_interlocked_signal_pattern}}\pysiglinewithargsret{\sphinxbfcode{\sphinxupquote{differential\_interlocked\_signal\_pattern}}}{\emph{\DUrole{n}{x\_start}}, \emph{\DUrole{n}{y\_start}}, \emph{\DUrole{n}{direction}}}{}
\sphinxAtStartPar
Generate the gcode to print one side (the signal side) of two interlocked patterns, one going up and one going down
\begin{quote}\begin{description}
\item[{Parameters}] \leavevmode\begin{itemize}
\item {} 
\sphinxAtStartPar
\sphinxstyleliteralstrong{\sphinxupquote{x\_start}} \textendash{} The x location of the bottom left corner of the pattern

\item {} 
\sphinxAtStartPar
\sphinxstyleliteralstrong{\sphinxupquote{y\_start}} \textendash{} The y location of the bottom left corner of the pattern

\item {} 
\sphinxAtStartPar
\sphinxstyleliteralstrong{\sphinxupquote{direction}} \textendash{} The direction the pattern should be printed in. Options are: ‘+y’,’+x’

\end{itemize}

\item[{Returns}] \leavevmode
\sphinxAtStartPar
The gcode to generate the pattern

\item[{Return type}] \leavevmode
\sphinxAtStartPar
string

\end{description}\end{quote}

\end{fulllineitems}

\index{effective\_length() (calibration\_pattern.calibration\_pattern method)@\spxentry{effective\_length()}\spxextra{calibration\_pattern.calibration\_pattern method}}

\begin{fulllineitems}
\phantomsection\label{\detokenize{index:calibration_pattern.calibration_pattern.effective_length}}\pysiglinewithargsret{\sphinxbfcode{\sphinxupquote{effective\_length}}}{}{}
\sphinxAtStartPar
Calculates actual length of a structure, taking into account that the
\begin{quote}\begin{description}
\item[{Returns}] \leavevmode
\sphinxAtStartPar
The actual length of a structure

\item[{Return type}] \leavevmode
\sphinxAtStartPar
float

\end{description}\end{quote}

\end{fulllineitems}

\index{effective\_length\_interlocked() (calibration\_pattern.calibration\_pattern method)@\spxentry{effective\_length\_interlocked()}\spxextra{calibration\_pattern.calibration\_pattern method}}

\begin{fulllineitems}
\phantomsection\label{\detokenize{index:calibration_pattern.calibration_pattern.effective_length_interlocked}}\pysiglinewithargsret{\sphinxbfcode{\sphinxupquote{effective\_length\_interlocked}}}{}{}
\sphinxAtStartPar
Calculates actual length of a structure, taking into account that the
\begin{quote}\begin{description}
\item[{Returns}] \leavevmode
\sphinxAtStartPar
The actual length of a structure

\item[{Return type}] \leavevmode
\sphinxAtStartPar
float

\end{description}\end{quote}

\end{fulllineitems}

\index{full\_interlocked\_print() (calibration\_pattern.calibration\_pattern method)@\spxentry{full\_interlocked\_print()}\spxextra{calibration\_pattern.calibration\_pattern method}}

\begin{fulllineitems}
\phantomsection\label{\detokenize{index:calibration_pattern.calibration_pattern.full_interlocked_print}}\pysiglinewithargsret{\sphinxbfcode{\sphinxupquote{full\_interlocked\_print}}}{\emph{\DUrole{n}{tool\_list}}, \emph{\DUrole{n}{reference\_tool}}, \emph{\DUrole{n}{save\_file\_name}}}{}
\sphinxAtStartPar
Generate the gcode to print a complete interlocked calibration pattern, that can be scanned and analysed to find the xy offsets.
\begin{quote}\begin{description}
\item[{Parameters}] \leavevmode\begin{itemize}
\item {} 
\sphinxAtStartPar
\sphinxstyleliteralstrong{\sphinxupquote{tool}} \textendash{} List of tool number of the tools. Each tool will be used for 4 calibration patterns. One in both the positive and negative x and y directions.

\item {} 
\sphinxAtStartPar
\sphinxstyleliteralstrong{\sphinxupquote{save\_file\_name}} \textendash{} Name of the file the gcode will be written to

\end{itemize}

\item[{Returns}] \leavevmode
\sphinxAtStartPar
The gcode to generate the print

\item[{Return type}] \leavevmode
\sphinxAtStartPar
string

\end{description}\end{quote}

\end{fulllineitems}

\index{gen (calibration\_pattern.calibration\_pattern attribute)@\spxentry{gen}\spxextra{calibration\_pattern.calibration\_pattern attribute}}

\begin{fulllineitems}
\phantomsection\label{\detokenize{index:calibration_pattern.calibration_pattern.gen}}\pysigline{\sphinxbfcode{\sphinxupquote{gen}}\sphinxbfcode{\sphinxupquote{\DUrole{w}{  }\DUrole{p}{=}\DUrole{w}{  }\textless{}generator.generator object\textgreater{}}}}
\sphinxAtStartPar
An instance of the generator class to generate the gcode

\end{fulllineitems}

\index{interlocked\_period (calibration\_pattern.calibration\_pattern attribute)@\spxentry{interlocked\_period}\spxextra{calibration\_pattern.calibration\_pattern attribute}}

\begin{fulllineitems}
\phantomsection\label{\detokenize{index:calibration_pattern.calibration_pattern.interlocked_period}}\pysigline{\sphinxbfcode{\sphinxupquote{interlocked\_period}}\sphinxbfcode{\sphinxupquote{\DUrole{w}{  }\DUrole{p}{=}\DUrole{w}{  }4}}}
\sphinxAtStartPar
How many milliemeters it takes before the structure repeats itself

\end{fulllineitems}

\index{interlocked\_pitch (calibration\_pattern.calibration\_pattern attribute)@\spxentry{interlocked\_pitch}\spxextra{calibration\_pattern.calibration\_pattern attribute}}

\begin{fulllineitems}
\phantomsection\label{\detokenize{index:calibration_pattern.calibration_pattern.interlocked_pitch}}\pysigline{\sphinxbfcode{\sphinxupquote{interlocked\_pitch}}\sphinxbfcode{\sphinxupquote{\DUrole{w}{  }\DUrole{p}{=}\DUrole{w}{  }0.75}}}
\sphinxAtStartPar
The pitch of the lines in the center of the structure in millimeters

\end{fulllineitems}

\index{interlocked\_reference\_pattern() (calibration\_pattern.calibration\_pattern method)@\spxentry{interlocked\_reference\_pattern()}\spxextra{calibration\_pattern.calibration\_pattern method}}

\begin{fulllineitems}
\phantomsection\label{\detokenize{index:calibration_pattern.calibration_pattern.interlocked_reference_pattern}}\pysiglinewithargsret{\sphinxbfcode{\sphinxupquote{interlocked\_reference\_pattern}}}{\emph{\DUrole{n}{x\_start}}, \emph{\DUrole{n}{y\_start}}, \emph{\DUrole{n}{direction}}}{}
\sphinxAtStartPar
Generate the gcode to print a single interlocked reference pattern
\begin{quote}\begin{description}
\item[{Parameters}] \leavevmode\begin{itemize}
\item {} 
\sphinxAtStartPar
\sphinxstyleliteralstrong{\sphinxupquote{x\_start}} \textendash{} The x location of the bottom left corner of the pattern

\item {} 
\sphinxAtStartPar
\sphinxstyleliteralstrong{\sphinxupquote{y\_start}} \textendash{} The y location of the bottom left corner of the pattern

\item {} 
\sphinxAtStartPar
\sphinxstyleliteralstrong{\sphinxupquote{direction}} \textendash{} The direction the pattern should be printed in. Options are: ‘+y’, ‘\sphinxhyphen{}y’,’+x’, ‘\sphinxhyphen{}x’

\end{itemize}

\item[{Returns}] \leavevmode
\sphinxAtStartPar
The gcode to generate the pattern

\item[{Return type}] \leavevmode
\sphinxAtStartPar
string

\end{description}\end{quote}

\end{fulllineitems}

\index{interlocked\_signal\_pattern() (calibration\_pattern.calibration\_pattern method)@\spxentry{interlocked\_signal\_pattern()}\spxextra{calibration\_pattern.calibration\_pattern method}}

\begin{fulllineitems}
\phantomsection\label{\detokenize{index:calibration_pattern.calibration_pattern.interlocked_signal_pattern}}\pysiglinewithargsret{\sphinxbfcode{\sphinxupquote{interlocked\_signal\_pattern}}}{\emph{\DUrole{n}{x\_start}}, \emph{\DUrole{n}{y\_start}}, \emph{\DUrole{n}{direction}}}{}
\end{fulllineitems}

\index{length (calibration\_pattern.calibration\_pattern attribute)@\spxentry{length}\spxextra{calibration\_pattern.calibration\_pattern attribute}}

\begin{fulllineitems}
\phantomsection\label{\detokenize{index:calibration_pattern.calibration_pattern.length}}\pysigline{\sphinxbfcode{\sphinxupquote{length}}\sphinxbfcode{\sphinxupquote{\DUrole{w}{  }\DUrole{p}{=}\DUrole{w}{  }70}}}
\sphinxAtStartPar
Total length of all the meanders

\end{fulllineitems}

\index{meander\_print() (calibration\_pattern.calibration\_pattern method)@\spxentry{meander\_print()}\spxextra{calibration\_pattern.calibration\_pattern method}}

\begin{fulllineitems}
\phantomsection\label{\detokenize{index:calibration_pattern.calibration_pattern.meander_print}}\pysiglinewithargsret{\sphinxbfcode{\sphinxupquote{meander\_print}}}{\emph{\DUrole{n}{tool}}, \emph{\DUrole{n}{save\_file\_name}}}{}
\sphinxAtStartPar
Generate the gcode to print a simple meandering structure, that fir example can be used to better understand conduction in 3D printed conductors.
\begin{quote}\begin{description}
\item[{Parameters}] \leavevmode\begin{itemize}
\item {} 
\sphinxAtStartPar
\sphinxstyleliteralstrong{\sphinxupquote{tool}} \textendash{} tool number of the tool that will be used for the print

\item {} 
\sphinxAtStartPar
\sphinxstyleliteralstrong{\sphinxupquote{save\_file\_name}} \textendash{} Name of the file the gcode will be written to

\end{itemize}

\item[{Returns}] \leavevmode
\sphinxAtStartPar
The gcode to generate the print

\item[{Return type}] \leavevmode
\sphinxAtStartPar
string

\end{description}\end{quote}

\end{fulllineitems}

\index{pitch (calibration\_pattern.calibration\_pattern attribute)@\spxentry{pitch}\spxextra{calibration\_pattern.calibration\_pattern attribute}}

\begin{fulllineitems}
\phantomsection\label{\detokenize{index:calibration_pattern.calibration_pattern.pitch}}\pysigline{\sphinxbfcode{\sphinxupquote{pitch}}\sphinxbfcode{\sphinxupquote{\DUrole{w}{  }\DUrole{p}{=}\DUrole{w}{  }1}}}
\sphinxAtStartPar
Milimeter spacing between the lines of the test pattern

\end{fulllineitems}

\index{repetitions() (calibration\_pattern.calibration\_pattern method)@\spxentry{repetitions()}\spxextra{calibration\_pattern.calibration\_pattern method}}

\begin{fulllineitems}
\phantomsection\label{\detokenize{index:calibration_pattern.calibration_pattern.repetitions}}\pysiglinewithargsret{\sphinxbfcode{\sphinxupquote{repetitions}}}{}{}
\sphinxAtStartPar
Calculates the number of repetions/periods of the repetitie pattern.
\begin{quote}\begin{description}
\item[{Returns}] \leavevmode
\sphinxAtStartPar
The number of repetions/periods of the repetitie pattern.

\item[{Return type}] \leavevmode
\sphinxAtStartPar
int

\end{description}\end{quote}

\end{fulllineitems}

\index{repetitions\_interlocked() (calibration\_pattern.calibration\_pattern method)@\spxentry{repetitions\_interlocked()}\spxextra{calibration\_pattern.calibration\_pattern method}}

\begin{fulllineitems}
\phantomsection\label{\detokenize{index:calibration_pattern.calibration_pattern.repetitions_interlocked}}\pysiglinewithargsret{\sphinxbfcode{\sphinxupquote{repetitions\_interlocked}}}{}{}
\sphinxAtStartPar
Calculates the number of repetions/periods of the interlocked repetitive pattern.
\begin{quote}\begin{description}
\item[{Returns}] \leavevmode
\sphinxAtStartPar
The number of repetions/periods of the interlocked repetitive pattern.

\item[{Return type}] \leavevmode
\sphinxAtStartPar
int

\end{description}\end{quote}

\end{fulllineitems}

\index{sigref\_only (calibration\_pattern.calibration\_pattern attribute)@\spxentry{sigref\_only}\spxextra{calibration\_pattern.calibration\_pattern attribute}}

\begin{fulllineitems}
\phantomsection\label{\detokenize{index:calibration_pattern.calibration_pattern.sigref_only}}\pysigline{\sphinxbfcode{\sphinxupquote{sigref\_only}}\sphinxbfcode{\sphinxupquote{\DUrole{w}{  }\DUrole{p}{=}\DUrole{w}{  }2}}}
\sphinxAtStartPar
Space where this only a sig or a ref pattern

\end{fulllineitems}

\index{single\_pattern() (calibration\_pattern.calibration\_pattern method)@\spxentry{single\_pattern()}\spxextra{calibration\_pattern.calibration\_pattern method}}

\begin{fulllineitems}
\phantomsection\label{\detokenize{index:calibration_pattern.calibration_pattern.single_pattern}}\pysiglinewithargsret{\sphinxbfcode{\sphinxupquote{single\_pattern}}}{\emph{\DUrole{n}{x\_start}}, \emph{\DUrole{n}{y\_start}}, \emph{\DUrole{n}{direction}}}{}
\sphinxAtStartPar
Generate the gcode to print a single simple reference pattern
\begin{quote}\begin{description}
\item[{Parameters}] \leavevmode\begin{itemize}
\item {} 
\sphinxAtStartPar
\sphinxstyleliteralstrong{\sphinxupquote{x\_start}} \textendash{} The x location of the bottom left corner of the pattern

\item {} 
\sphinxAtStartPar
\sphinxstyleliteralstrong{\sphinxupquote{y\_start}} \textendash{} The y location of the bottom left corner of the pattern

\item {} 
\sphinxAtStartPar
\sphinxstyleliteralstrong{\sphinxupquote{direction}} \textendash{} The direction the pattern should be printed in. Options are: ‘+y’, ‘\sphinxhyphen{}y’,’+x’, ‘\sphinxhyphen{}x’

\end{itemize}

\item[{Returns}] \leavevmode
\sphinxAtStartPar
The gcode to generate the pattern

\item[{Return type}] \leavevmode
\sphinxAtStartPar
string

\end{description}\end{quote}

\end{fulllineitems}

\index{spacing (calibration\_pattern.calibration\_pattern attribute)@\spxentry{spacing}\spxextra{calibration\_pattern.calibration\_pattern attribute}}

\begin{fulllineitems}
\phantomsection\label{\detokenize{index:calibration_pattern.calibration_pattern.spacing}}\pysigline{\sphinxbfcode{\sphinxupquote{spacing}}\sphinxbfcode{\sphinxupquote{\DUrole{w}{  }\DUrole{p}{=}\DUrole{w}{  }3}}}
\sphinxAtStartPar
Spacing between two patterns of different nozzles

\end{fulllineitems}

\index{spacing\_to\_square (calibration\_pattern.calibration\_pattern attribute)@\spxentry{spacing\_to\_square}\spxextra{calibration\_pattern.calibration\_pattern attribute}}

\begin{fulllineitems}
\phantomsection\label{\detokenize{index:calibration_pattern.calibration_pattern.spacing_to_square}}\pysigline{\sphinxbfcode{\sphinxupquote{spacing\_to\_square}}\sphinxbfcode{\sphinxupquote{\DUrole{w}{  }\DUrole{p}{=}\DUrole{w}{  }5}}}
\sphinxAtStartPar
Spacing between the patterns and the square

\end{fulllineitems}

\index{square\_lines (calibration\_pattern.calibration\_pattern attribute)@\spxentry{square\_lines}\spxextra{calibration\_pattern.calibration\_pattern attribute}}

\begin{fulllineitems}
\phantomsection\label{\detokenize{index:calibration_pattern.calibration_pattern.square_lines}}\pysigline{\sphinxbfcode{\sphinxupquote{square\_lines}}\sphinxbfcode{\sphinxupquote{\DUrole{w}{  }\DUrole{p}{=}\DUrole{w}{  }3}}}
\sphinxAtStartPar
Number of lines of the square around the structure

\end{fulllineitems}

\index{square\_pattern() (calibration\_pattern.calibration\_pattern method)@\spxentry{square\_pattern()}\spxextra{calibration\_pattern.calibration\_pattern method}}

\begin{fulllineitems}
\phantomsection\label{\detokenize{index:calibration_pattern.calibration_pattern.square_pattern}}\pysiglinewithargsret{\sphinxbfcode{\sphinxupquote{square\_pattern}}}{\emph{\DUrole{n}{x\_start}}, \emph{\DUrole{n}{y\_start}}, \emph{\DUrole{n}{x}}, \emph{\DUrole{n}{y}}, \emph{\DUrole{n}{clockwise}}, \emph{\DUrole{n}{n}}}{}
\sphinxAtStartPar
Generate the gcode to print a square at a specific location with a specific size and a specified thickness
\begin{quote}\begin{description}
\item[{Parameters}] \leavevmode\begin{itemize}
\item {} 
\sphinxAtStartPar
\sphinxstyleliteralstrong{\sphinxupquote{x\_start}} \textendash{} The x location of the bottom left corner of the square

\item {} 
\sphinxAtStartPar
\sphinxstyleliteralstrong{\sphinxupquote{y\_start}} \textendash{} The y location of the bottom left corner of the square

\item {} 
\sphinxAtStartPar
\sphinxstyleliteralstrong{\sphinxupquote{x}} \textendash{} The size of the square in the x direction

\item {} 
\sphinxAtStartPar
\sphinxstyleliteralstrong{\sphinxupquote{y}} \textendash{} The size of the square in the y direction

\item {} 
\sphinxAtStartPar
\sphinxstyleliteralstrong{\sphinxupquote{clockwise}} \textendash{} Boolean indicating if the square should be printed clockwise or counter clockwise

\item {} 
\sphinxAtStartPar
\sphinxstyleliteralstrong{\sphinxupquote{n}} \textendash{} The thickness of the lines of the square in number of times the nozzle diameter

\end{itemize}

\item[{Returns}] \leavevmode
\sphinxAtStartPar
The gcode to generate the square

\item[{Return type}] \leavevmode
\sphinxAtStartPar
string

\end{description}\end{quote}

\end{fulllineitems}

\index{total\_height() (calibration\_pattern.calibration\_pattern method)@\spxentry{total\_height()}\spxextra{calibration\_pattern.calibration\_pattern method}}

\begin{fulllineitems}
\phantomsection\label{\detokenize{index:calibration_pattern.calibration_pattern.total_height}}\pysiglinewithargsret{\sphinxbfcode{\sphinxupquote{total\_height}}}{}{}
\sphinxAtStartPar
Calculates the total height of the patterns in both directions together
\begin{quote}\begin{description}
\item[{Returns}] \leavevmode
\sphinxAtStartPar
the total height of the patterns in both directions together

\item[{Return type}] \leavevmode
\sphinxAtStartPar
float

\end{description}\end{quote}

\end{fulllineitems}

\index{total\_height\_interlocked() (calibration\_pattern.calibration\_pattern method)@\spxentry{total\_height\_interlocked()}\spxextra{calibration\_pattern.calibration\_pattern method}}

\begin{fulllineitems}
\phantomsection\label{\detokenize{index:calibration_pattern.calibration_pattern.total_height_interlocked}}\pysiglinewithargsret{\sphinxbfcode{\sphinxupquote{total\_height\_interlocked}}}{}{}
\sphinxAtStartPar
Calculates the total height of the patterns in both directions together in case of an interlocked print
\begin{quote}\begin{description}
\item[{Returns}] \leavevmode
\sphinxAtStartPar
the total height of the patterns in both directions together

\item[{Return type}] \leavevmode
\sphinxAtStartPar
float

\end{description}\end{quote}

\end{fulllineitems}

\index{total\_one\_dir\_width() (calibration\_pattern.calibration\_pattern method)@\spxentry{total\_one\_dir\_width()}\spxextra{calibration\_pattern.calibration\_pattern method}}

\begin{fulllineitems}
\phantomsection\label{\detokenize{index:calibration_pattern.calibration_pattern.total_one_dir_width}}\pysiglinewithargsret{\sphinxbfcode{\sphinxupquote{total\_one\_dir\_width}}}{}{}
\sphinxAtStartPar
Calculates the total width of all the patterns in one direction
\begin{quote}\begin{description}
\item[{Returns}] \leavevmode
\sphinxAtStartPar
the total width of all the patterns in one direction

\item[{Return type}] \leavevmode
\sphinxAtStartPar
float

\end{description}\end{quote}

\end{fulllineitems}

\index{total\_width() (calibration\_pattern.calibration\_pattern method)@\spxentry{total\_width()}\spxextra{calibration\_pattern.calibration\_pattern method}}

\begin{fulllineitems}
\phantomsection\label{\detokenize{index:calibration_pattern.calibration_pattern.total_width}}\pysiglinewithargsret{\sphinxbfcode{\sphinxupquote{total\_width}}}{}{}
\sphinxAtStartPar
Calculates the total width of the patterns in both directions together
\begin{quote}\begin{description}
\item[{Returns}] \leavevmode
\sphinxAtStartPar
the total width of the patterns in both directions together

\item[{Return type}] \leavevmode
\sphinxAtStartPar
float

\end{description}\end{quote}

\end{fulllineitems}

\index{total\_width\_interlocked() (calibration\_pattern.calibration\_pattern method)@\spxentry{total\_width\_interlocked()}\spxextra{calibration\_pattern.calibration\_pattern method}}

\begin{fulllineitems}
\phantomsection\label{\detokenize{index:calibration_pattern.calibration_pattern.total_width_interlocked}}\pysiglinewithargsret{\sphinxbfcode{\sphinxupquote{total\_width\_interlocked}}}{}{}
\sphinxAtStartPar
Calculates the total width of the patterns in both directions together in case of an interlocked print
\begin{quote}\begin{description}
\item[{Returns}] \leavevmode
\sphinxAtStartPar
the total width of the patterns in both directions together

\item[{Return type}] \leavevmode
\sphinxAtStartPar
float

\end{description}\end{quote}

\end{fulllineitems}

\index{width (calibration\_pattern.calibration\_pattern attribute)@\spxentry{width}\spxextra{calibration\_pattern.calibration\_pattern attribute}}

\begin{fulllineitems}
\phantomsection\label{\detokenize{index:calibration_pattern.calibration_pattern.width}}\pysigline{\sphinxbfcode{\sphinxupquote{width}}\sphinxbfcode{\sphinxupquote{\DUrole{w}{  }\DUrole{p}{=}\DUrole{w}{  }8}}}
\sphinxAtStartPar
Length of lines in the test pattern

\end{fulllineitems}


\end{fulllineitems}



\chapter{Indices and tables}
\label{\detokenize{index:indices-and-tables}}\begin{itemize}
\item {} 
\sphinxAtStartPar
\DUrole{xref,std,std-ref}{genindex}

\item {} 
\sphinxAtStartPar
\DUrole{xref,std,std-ref}{search}

\end{itemize}


\renewcommand{\indexname}{Python Module Index}
\begin{sphinxtheindex}
\let\bigletter\sphinxstyleindexlettergroup
\bigletter{c}
\item\relax\sphinxstyleindexentry{calibartion\_pattern}\sphinxstyleindexpageref{index:\detokenize{module-calibartion_pattern}}
\item\relax\sphinxstyleindexentry{calibration\_pattern}\sphinxstyleindexpageref{index:\detokenize{module-calibration_pattern}}
\indexspace
\bigletter{g}
\item\relax\sphinxstyleindexentry{generator}\sphinxstyleindexpageref{index:\detokenize{module-0}}
\end{sphinxtheindex}

\renewcommand{\indexname}{Index}
\printindex
\end{document}